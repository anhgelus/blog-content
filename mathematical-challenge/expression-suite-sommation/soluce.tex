\documentclass[a4paper,fleqn]{article}

\usepackage{bbm}
\usepackage{amsmath}
\usepackage{amssymb}
\date{\today}

\begin{document}
Je vous laisse le soin de calculer les premiers termes de $x_n$ par vous mêmes. \\
On en déduit la conjecture suivante :
\[
	x_{n+1}=2^n
\]
Démontrons le par récurrence. \\
\\ \\
Soit $P_n$ une propriété telle que :
\[
	P_n:~ x_{n+1}=2^n
\]
Montrons que cette propriété est vraie pour tout $n$ appartenant à $\mathbb{N}$. \\ \\
$\textit{Initialisation}$~ Calculons $2_0$.
\[
	2^0 ~=~ 1 ~\textit{Par convention} 
\]
Il s'agit exactement de $x_0$. Ainsi $P_n$ est initialisée. \\ \\
$\textit{Hérédité}$~ Fixons $n$ dans $\mathbb{N}$ tel que $P_n$ soit vraie.
Alors :
\begin{alignat*}{3}
	&&x_{n+1}&&~=~&2^n \\
	&\Leftrightarrow&x_{n+1}&&~=~&\displaystyle\sum^n_{k=0}x_k \\
	&\Leftrightarrow&x_{n+2}&&~=~&\displaystyle\sum^n_{k=0}(x_k)+x_{n+1} \\
	&\Leftrightarrow&x_{n+2}&&~=~&\displaystyle\sum^n_{k=0}(x_k)+\displaystyle\sum^n_{k=0}(x_k) \\
	&\Leftrightarrow&x_{n+2}&&~=~&2\displaystyle\sum^n_{k=0}(x_k) \\
	&\Leftrightarrow&x_{n+2}&&~=~&2\times 2^n \\
	&\Leftrightarrow&x_{n+2}&&~=~&2\times 2^{n+1} \\
\end{alignat*}
Ainsi $P_n$ est héréditaire. \\
Comme $P_0$ est vraie et que $P_n$ est héréditaire, alors, par principe de récurrence, $P_n$ est vraie pour tout $n$ appartenant à $\mathbb{N}$.
\end{document}

