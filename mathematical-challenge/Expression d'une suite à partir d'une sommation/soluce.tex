\documentclass[a4paper,fleqn]{article}
\usepackage{amsmath}
\usepackage{amsfonts}

\begin{document}
$\text{Soit } n \in \mathbb{N}$
\begin{align*}
    \sum_{k=0}^{n} u_k                         & = \frac{n^2+n}{3}                                                      \\
                                               & = \frac{n(n+1)}{3}\textit{ Par distributivité du produit sur la somme} \\ \\
    \textit{Or on sait que:}~~\sum_{k=0}^{n} k & = \frac{n(n+1)}{2}\textit{. Essayons de s'y ramener.}                  \\ \\
\end{align*}
\vspace{-1cm}
\begin{alignat*}{3}
     &                  & \sum_{k=0}^{n} u_k                   &  & ~=~ & \frac{2}{2}\times\frac{n (n+1)}{3}                                       \\
     & \Leftrightarrow~ & \sum_{k=0}^{n} u_k                   &  & ~=~ & \frac{2}{3}\times\frac{n(n+1)}{2}~\textit{ Par commutativité du produit} \\
     & \Leftrightarrow~ & \sum_{k=0}^{n} u_k                   &  & ~=~ & \frac{2}{3}\times\sum_{k=0}^{n} k                                        \\
     & \Leftrightarrow~ & \frac{3}{2}\times\sum_{k=0}^{n} u_k  &  & ~=~ & \sum_{k=0}^{n} k                                                         \\
     & \Leftrightarrow~ & \sum_{k=0}^{n} \frac{3}{2}\times u_k &  & ~=~ & \sum_{k=0}^{n} k~\textit{ Par distributivité du produit sur la somme}    \\
     & \Leftrightarrow~ & \frac{3}{2}\times u_k                &  & ~=~ & k,~\forall k \in \mathbb{N}                                              \\
     & \Leftrightarrow~ & u_n                                  &  & ~=~ & \frac{2}{3}\times n\textit{ Par changement d'indice (et pour n entier)}  \\
\end{alignat*}


\end{document}