\documentclass[a4paper,fleqn]{article}

\usepackage{bbm}
\usepackage{amsmath}
\usepackage{amsfonts}
\usepackage{amssymb}
\date{\today}
\newcommand{\mycomment}[1]{}

\begin{document}
	$\text{Soit } f(x) = -x^2-5x+6$ \\ \\
	Ainsi :
	\[
		\ln(-x^2-5x+6)=\ln(f(x))
	\]
	Vu que la fonction $\ln$ est définie sur $\mathbb{R}+^*$, il faut donc trouver quand $f$ n'est pas dans cet interval. \\
	Cela revient à résoudre :
	\[
		f(x)>0
	\]
	Résolvons :
	\begin{alignat*}{3}
		~ & f(x) & > & 0 \\
		\Leftrightarrow  & -x^2-5x+6        & ~>~ & 0 \\
		\Leftrightarrow  & -x^2-x+6x+6      & ~>~ & 0 \\
		\Leftrightarrow  & -(x(x-1)+6(x-1)) & ~>~ & 0 \\
		\Leftrightarrow  & -(x-1)(x+6)      & ~>~ & 0 \\
	\end{alignat*}
	L'ensemble solution est alors évident. \\
	$\text{S } = ]-6;1[$
\end{document}

